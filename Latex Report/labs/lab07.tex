\subsection{Упражнение 1}

Необходимо реализовать алгоритм БПФ. Для этого разделим массив сигнала на четные и нечетные элементы и затем вычислить ДФТ для обоих групп.
Также используем лемму Дэниеолсона-Ланцоша.

Для тестирования возьмем небольшой массив сигнала

\begin{lstlisting}[language=Python]
ys = [0.8, 0.7, -0.5, 0.5]
hs = np.fft.fft(ys)
hs
\end{lstlisting}
\begin{lstlisting}
array([ 1.5+0.j ,  1.3-0.2j, -0.9+0.j ,  1.3+0.2j])
\end{lstlisting}

Применим ДФТ функцию, которая представлена в предыдущем пункте, то есть в блокноте к главе книги.

\begin{lstlisting}[language=Python]
hs2 = dft(ys)
np.sum(np.abs(hs - hs2))
\end{lstlisting}
\begin{lstlisting}
7.249538831146999e-16
\end{lstlisting}

Теперь реализуем БПФ без рекурсии при помощи разделения элементов.

\begin{lstlisting}[language=Python]
def fft(ys):
  He = dft(ys[::2])
  Ho = dft(ys[1::2])
  ns = np.arange(len(ys))
  W = np.exp(-1j * PI2 * ns / len(ys))
  return np.tile(He, 2) + W * np.tile(Ho, 2)
\end{lstlisting}

\begin{lstlisting}[language=Python]
fft(ys)
\end{lstlisting}
\begin{lstlisting}
array([ 1.5+0.j ,  1.3-0.2j, -0.9-0.j ,  1.3+0.2j])
\end{lstlisting}

Теперь реализуем вариант алгоритма с рекурсией

\begin{lstlisting}[language=Python]
def fft_rec(ys):
  if len(ys) == 1:
    return ys
  He = fft_rec(ys[::2])
  Ho = fft_rec(ys[1::2])
  ns = np.arange(len(ys))
  W = np.exp(-1j * PI2 * ns / len(ys))
  return np.tile(He, 2) + W * np.tile(Ho, 2)
\end{lstlisting}

\begin{lstlisting}[language=Python]
fft_rec(ys)
\end{lstlisting}
\begin{lstlisting}
array([ 1.5+0.j ,  1.3-0.2j, -0.9-0.j ,  1.3+0.2j])
\end{lstlisting}

\subsection{Вывод}

В данной лабораторной работе был реализован алгоритм БПФ и протестирован, алгоритм работает в точности также, как и библиотечная функция. 
